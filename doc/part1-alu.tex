\documentclass[a4paper]{article} 

% deutsche Sonderzeichen benutzen
% deutsche Trennregel
\usepackage[german]{babel}

% wegen deutschen Umlauten
\usepackage[utf8]{inputenc}

\usepackage[T1]{fontenc}
% hat was mit Abstaenden zu tun
%\frenchspacing

%\usepackage{subfigure}


\usepackage{dashrule}
\usepackage{caption, subcaption}
\usepackage{calc}
\usepackage{multicol}

\usepackage{amsmath}
\usepackage{amsfonts}
%multiple rows in table
\usepackage{multirow}
\usepackage{latexsym}

\usepackage{tikz}
\usepackage{tikz-qtree}
%\usetikzlibrary{shapes, arrows, decorations.pathmorphing,circuits}
\usepackage{tikz-timing}
% Truth table produced by http://www.kwi.dk/projects/php/truthtable/? and modified
\usetikzlibrary{trees}

% haleluia: Palatino Font
\usepackage{palatino}

\usepackage[absolute]{textpos}
\setlength{\TPHorizModule}{30mm}
\setlength{\TPVertModule}{\TPHorizModule}
% start everything near the top-left corner
\textblockorigin{10mm}{10mm}


%we very much like \xhookleftarrow
\usepackage{empheq}

%we very much like listings
\usepackage{listings}

%we very much like graphics, oh yes
%\usepackage{graphicx}

% dirtree tikz mode
\makeatletter
\newcount\dirtree@lvl
\newcount\dirtree@plvl
\newcount\dirtree@clvl
\def\dirtree@growth{%
  \ifnum\tikznumberofcurrentchild=1\relax
  \global\advance\dirtree@plvl by 1
  \expandafter\xdef\csname dirtree@p@\the\dirtree@plvl\endcsname{\the\dirtree@lvl}
  \fi
  \global\advance\dirtree@lvl by 1\relax
  \dirtree@clvl=\dirtree@lvl
  \advance\dirtree@clvl by -\csname dirtree@p@\the\dirtree@plvl\endcsname
  \pgf@xa=1cm\relax
  \pgf@ya=-1cm\relax
  \pgf@ya=\dirtree@clvl\pgf@ya
  \pgftransformshift{\pgfqpoint{\the\pgf@xa}{\the\pgf@ya}}%
  \ifnum\tikznumberofcurrentchild=\tikznumberofchildren
  \global\advance\dirtree@plvl by -1
  \fi
}

\tikzset{
  dirtree/.style={
    growth function=\dirtree@growth,
    every node/.style={anchor=north},
    every child node/.style={anchor=west},
    edge from parent path={(\tikzparentnode\tikzparentanchor) |- (\tikzchildnode\tikzchildanchor)}
  }
}
\makeatother


\usepackage{fancyhdr}                     %Declares the package fancyhdr
\pagestyle{fancy}                         %Forces the page to use the fancy template
\fancyhead[L]{GInf Project Part ALU}
\fancyhead[R]{26.11.2013}
\fancyfoot[L]{\includegraphics[width=\textwidth]{../../pics/students.png}}
\fancyfoot[C]{}

% \fancyfoot[L]{\begin{picture}(3,2)(80,80)
% %\IfFileExists{Tex-Tmpl.png} 
% {
% \includegraphics[scale=0.2]{../Logo_Berner_Fachhochschule.pdf}}{}
% \end{picture}}

\newcommand{\textline}{\vspace*{1cm}\dotfill}

% provides hooks to do actions on every page
\usepackage{everypage}

%\usepackage{algorithmic}
%\usepackage{algorithm}
\usepackage{listings}



% \newcommand*\BFHLogo{%
%   \tikz{%[remember picture,overlay] {%
%     % \draw[thick,red, decorate,decoration={snake}]
%     % (current page.north west) rectangle (current page.south east);
%     \node (current page.south west) {\includegraphics[scale=0.4]{../Logo_Berner_Fachhochschule.pdf}} ;
%   }%
% }
\usepackage{graphicx}

\begin{document} 
%
%
%\AddEverypageHook{\BFHLogo}




\begin{center}
  \Large \textsc{\underline{ALU: Aritmetic \& Logic Unit}}
\end{center}

\begin{description}
\item[Goal:] Implement all empty function-bodies in \texttt{alu.c}. Don't change anything outside the function bodies
\item[File layout:]\qquad\\
\tikzstyle{every node}=[draw=black,thick,anchor=west]
\tikzstyle{selected}=[draw=red,fill=red!30]
\tikzstyle{optional}=[dashed,fill=gray!50]

\begin{tikzpicture}[%
  grow via three points={one child at (0.5,-0.7) and
  two children at (0.5,-0.7) and (0.5,-1.4)},
  edge from parent path={(\tikzparentnode.south) |- (\tikzchildnode.west)}]
  \node {src}
    child { node {include}}		
    child { node {memory}}
    child { node {register}}
    child { node [selected] {alu}
      child { node {alu-main.c}}
      child { node [optional] {alu.c}}
    }
    child [missing] {}				
    child [missing] {}				
    child [missing] {}				
    child { node {more stuff}};
\end{tikzpicture}

%  \begin{tikzpicture}[grow=right]
%  \tikzset{level distance=60pt,sibling distance=18pt}
% % \tikzset{every leaf node/.style=\it}
%   \tikzset{execute at begin node=\strut}
% %  \Tree [.<project root> [.src  [.alu alu-main.c alu.c [.alu sdf s df ] [.alu-main.c ] ] ] ]
% %  \Tree [.<project root> [.src   [.alu alu-main.c alu.c  ]  ] ]
  
  
%   \tikzset{frontier/.style={distance from root=250pt}}
%   \Tree 
%   [.<project\ root>   [.src [.include alu.h alu-opcodes.h ] makefile  [.alu \textbf{alu.c} alu-main.c ] ] ]
 

% %[.include alu.h alu-opcodes.h ]
% \end{tikzpicture}

\item[Compiling:] \texttt{make alu-main} compiles \textbf{alu-main}
\item[Read-eval-loop:] \texttt{./alu-main} gives you an empty prompt. The program is reading from  \texttt{stdio} , hence waiting for input from your keyboard. A \texttt{Ctrl-D} simulates an EOF (End-of-file) and the program quits. While reading form the \texttt{stdio} you can enter any of these commands:
  \begin{itemize}
  \item \texttt{reset}
  \item \texttt{add <HH> <HH>}
  \item \texttt{sub <HH> <HH>}
  \item \texttt{or <HH> <HH>}
  \item \texttt{and <HH> <HH>}
  \item \texttt{xor <HH> <HH>}
  \item \texttt{neg\_a <HH>}
  \item \texttt{neg\_b <HH>}
  \item \texttt{not\_a <HH>}
  \item \texttt{not\_b <HH>}
  \end{itemize}
\item[Testing:] \texttt{make alu-test} compiles, runs and tests the alu
\item[Time:] Two weeks, i.e. 10.12.2012 at the beginning of the lecture you give in a print-out from \texttt{alu.c}.
\item[Delivery:] \texttt{alu.c} \emph{must be printed two-on-one page, with pretty-print} (use a2ps or enscript, look into makefile for examples)
\end{description}

\end{document}